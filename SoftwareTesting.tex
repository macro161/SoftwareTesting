\documentclass[oneside]{VUMIFPSkursinis}
\usepackage{algorithmicx}
\usepackage{algorithm}
\usepackage{algpseudocode}
\usepackage{amsfonts}
\usepackage{float}
\usepackage{amsmath}
\usepackage{bm}
\usepackage{caption}
\usepackage{color}
\usepackage{float}
\usepackage{graphicx}
\usepackage{listings}
\usepackage{subfig}
\usepackage{ltablex}
\usepackage{longtable}
\usepackage{wrapfig}
\usepackage{subfig}
\usepackage{pbox}
\renewcommand{\labelenumii}{\theenumii}
\renewcommand{\theenumii}{\theenumi.\arabic{enumii}.}
\renewcommand{\labelenumiii}{\theenumiii}
\renewcommand{\theenumiii}{\theenumii\arabic{enumiii}.}
\newcolumntype{P}[1]{>{\centering\arraybackslash}p{#1}}
\usepackage[%
	colorlinks=true,
	linkcolor=black
]{hyperref}
\university{Vilniaus universitetas}
\faculty{Matematikos ir informatikos fakultetas}
\department{Programų sistemų katedra}
\papertype{Programų sistemų testavimo laboratorinis darbas}
\title{Internetinės aplikacijos Cronometer testavimas}
\titleineng{Web appliaction Cronometer testing}
\status{3 kurso studentas}
\author{Matas Savickis}


\supervisor{Vytautas Valaitis, Asist., Dr.}
\date{Vilnius – \the\year}

\bibliography{bibliografija}
\begin{document}
\maketitle

\sectionnonum{Anotacija}
Šiuo darbu yra siekiama sukurti testavimo planą, testavimo atvejus ir atlikti testavimą www.cronometer.com  internetinei aplikacijai skirtai sekti maisto informaciją, sporto aktyvumą ir biologinius rodiklius.

\begin{itemize}
	\item{Matas Savickis - savickis.matas@gmail.com}
\end{itemize}

\tableofcontents


\sectionnonum{Testavimo planas}
\section{Testavimo plano indentifikatoriai} CMT-1.1.1 - pirmas skaičius rodo ar testavimas buvo atliktas telefonu ar kompiuterius(1 - kompiuteris 2 - telefonas) antras skaičius rodo kelinta funkcionalumą testuosime, trečias skaičius rodo kelintą scenarijų testuojame. 3.X.X naudosime visiems kitiems su vienu konkreciu funkcionalumu nesusijusiais testais
\section{Nuorodos}
	\begin{itemize}
		\item{https://jmpovedar.files.wordpress.com/2014/03/ieee-829.pdf}
		\item{http://pst.valaitis.net/PST}
	\end{itemize}
\section{Įvadas}
	Šis testavimo planas yra skirtas internetinės aplikacijos Cronometer testavimui. Pagrindinis testavimo tikslas bus įsitikinti ar su skirtingais įvesties duomenimis programa veikia korektiškai ir gražina korektiškus rezultatus. Bus testuojama maisto pridėjimas, sporto pridėjimas, biometrinių duomenų pridėjimas ir prisijungimas prie tinklalapio. Visas testavimas vyks rankiniu būdu.

\section{Testavimo objektai}
	\begin{enumerate}
		\item{www.cronometer.com}
	\end{enumerate}

\section{Rizikos}
	\begin{itemize}
		\item{Nėra programos dokumentacijos}
		\item{Jeigu kažką užlaušiu ant manęs pyks Cronometer kūrėjai}
	\end{itemize}

\section{Funkcionalumas kurį testuosime}
	\begin{enumerate}
		\item{Registravimasis į tinklalapi}
		\item{Prisijungimas prie tinklalapio}
		\item{Maisto vienetų pridėjimas sekimui, įvesties ir išvesties korektiškumas}
		\item{Sporto sekimo pridėjimas sekimui, įvesties ir išvesties korektiškumas}
		\item{Biometrinių duomenų pridėjimas į sistemą, įvesties ir išvesties korektiškumas}
	\end{enumerate}
\section{Funkcionalumas kurio netestuosime}
	\begin{enumerate}
		\item{Premium narystės funkcionalumus - kadangi testavimą atliks biednas studentas kuris neturi pinigų nusipirkti Premium narystės todėl šio funkcionalumo ir netestuosime}
		\item{Sistemos suderinamumą su išnamiaisiais įrenginiais sekimui(Fitbit) - priežastis ta pati kaip ir Premium narystės atveju}
		\item{Profilio funkcija - funkcionalumas nekritinis}
	\end{enumerate}	
\section{Strategija}
	\begin{itemize}
		\item{Specialūs įrankiai - nėra}
		\item{Apmokymas naudotis įrankiais - nėra}
		\item{Metrikos - bus sekama kiek laiko užtruko atlikti testavimą, suranstų klaidų skaičius, teisignai praeitų testų skaičius}
		\item{Metrikos bus sekamos atliekant rankinį testavimą}
		\item{Konfigūracijos valdymas - nėra}
		\item{Bus testuojama tik pradinė konfigūracija, kurią pateikia kūrėjai}
		\item{Bus naudojama stacionarus kompiuteris ir išmanusis telefonas}
		\item{Bus naudojama Google Chrome naršyklė abiem atvejais(PC ir telefonas)}
		\item{Regresiniai testai - nėra}
	\end{itemize}

\section{Sėkmės ir nesekmės kriterijai}
Testą laikysime pasisekusiu jeigu įvedus duomenis ar padarius veiksmą programa pateiks korektišką ir tikėtasi rezultatą ar neparodys klaidos pranešimų. Nepasisekusiu testu laikysime jeigu atlikus veiksą arba įvedus duomenis programos išvestis bus nenumatyta, netiksli, ne tai ko tikėjome arba programa gražins klaidos pranešimą.

\section{Testo nutraukimo criterijai}
Testavimą nutraukime tuo atveju jeigu su panašia įvestimi programa gražina tą patį rezultatą. Jeigu to pačio funkcionalumo varijacijoje(pvz. Maisto produktų dėjimas į sąrašą) klaida liks, laikysime, kad testas daugiau informacijos mums nebesuteiks ir testavimą nutrauksime įvertinti savo testavimo atvejams ir tolimesniam testavimui.

\section{Testų planai}
Planas skirtas rankiniam testavimui juodosios dežes metodu

\section{Likusios užduotys}
\begin{center}
\begin{tabular}{ |c|c|c| } 
 \hline
 Užduotis & Vykdytojas & Būsena \\ 
 Sukurti testavimo planą & Matas Savickis & Pabaigta \\ 
 Maisto pridėjimo testavimo atvejai  & Matas Savickis & Pabaigta \\ 
 Sporto pridėjimo testavimo atvejai & Matas Savickis & Pabaigta \\
 Biometrikos pridėjimo testavimo atvejai & Matas Savickis & Pabaigta \\
 \hline
\end{tabular}
\end{center}

\section{Aplinkos reikalavimai}
	\begin{itemize}
		\item{Turi veikti www.cronometer.com tinklalapios}
		\item{Reikalinga elektra}
		\item{Reikalingas interneto ryšys}
		\item{Atlikti testavimui reikia išmanioji telefono}
		\item{Atlikti testavimui reikia stacionaraus kompiuterio}
		\item{Norint pasiekti tiklalapį reikia modernios interneto naršyklės}
	\end{itemize}

\section{Personalas ir reikalingi apmokymai}
`	Testavimas bus atliekamas programų sistemų studentų. Komandoje iš viso puse vienas testuotojas. Svarbu kad testuotojas mokėtų naudotis kompiuterio pele bei klaviatūra. Taip pat svarbu, kad testuotojas neturėtų motorinių kurie keltų sunkumų naudotis telefono, pele ar klaviatūra. Taip pat yra svarbu, kad testuotojas mokėtų naudoti išmaniaisiais telefonais.

\section{Atsakomybės}
\begin{center}
\begin{tabular}{ |c|c| } 
 \hline
 Atsakomybė & Matas Savickis \\
 Rankinių testų kūrimas & X \\
 Rankinių testų vykdymas & X \\
 Testavimo plano sudarymas & X \\
 \hline
\end{tabular}
\end{center}

Už visą testavimo planavimo, sudarymo ir vykdymo procesą bus atskingas neapmokamas programų sistemų studentas Matas Savickis.  

\section{Tvarkaraštis}
Testavimo plano ir testavimo atvejų sudarymas ir testų atlikimas bus įvykdytas pagal gerai žinomą studentišką laiko valdymo metodą pavadinimu ,,viską padarysiu per vieną naktį prieš atsiskaitymą". Tai yra yra pats populiariausias darbo pasiskirstymo metodas, o kaip mes visi žinome tai kas populiaru yra gerai.\newline



Darbas bus atliekamas tokiais žingsniais:

\begin{enumerate}
	\item{Testavimo plano sudarymas}
	\item{Testavimo atvejų dokumento sudarymas}
	\item{Testavimo atvejų internetinei versijai sudarymas}
	\item{Testavimo atvejų mobiliai versijai sudarymas}
	\item{Internetinės versijos testų vykdymas}
	\item{Mobilios versijos testų vykdymas}
	\item{Rezultatų ir išvadų sudarymas}
\end{enumerate}

\section{Rizikos}
Testavimas gali būti nesėkmingas dėl testuotojų motyvacijos stokos. Kadangi darbas yra neapmokamas, darbai dažniausiai būna atidedami iki paskutinės minutės, kas gali sukelti problemų laiku spėjant pristatyti atliktus darbus. Rizika gali kirti ir dėl to, nes  testuotojas neturės sistemos dokumentacijos ir testo atvejai nebus pilni arba korektiški. Dėl dokumentacijos trūkumo gali nepavykti sugalvoti visų testavimo atvejų.

\sectionnonum{Testavimo atvejai}

\section{Įvadas}
Šioje dalyje bus pateikti Cronometer internetinės aplikacijos testavimo scenarijai.Šiuo darbu siekiama įsitikinti, kad pagrindiniai aplikacijos funkcionalumai veikia korektiškai tiek naudojantis internetine naršykle tiek mobiliuoju telefonu. Bus testuojamos šios dalys
	\begin{enumerate}
		\item{Maisto pridėjimo funkcija internetinėje aplikacijoje}
		\item{Sporto pridėjimo funkcija internetinėje aplikacijoje}
		\item{Biometrinių duomenų pridėjimas internetinėje aplikacijoje}
		\item{Maisto pridėjimo funkcija mobiliojoje aplikacijoje}
		\item{Sporto pridėjimo funkcija mobiliojoje aplikacijoje}
		\item{Biometrinių duomenų pridėjimas mobiliojoje aplikacijoje}
	\end{enumerate}

\section{Testavimo atvejai}

\iffalse XXXXXXXXXXXXXXXXXXXXXXXXXXXXXXXXXXXXXXXXXXXXXXXXXXXXXXXXXXXXXXXXXXXXXXXXXXXXXXXXXXXXXXXXXXXXXXXXXXXXXXXXXXXXXXXXXXXXXXXXXXXXXXXXXXXXXXX \fi
\begin{center}
    \begin{tabular}{ |p{5cm}|p{13cm}|}
    \hline
    	Testavimo atvejo numeris & CMT-1.1.1\\ \hline
    	Testavimo atvejis & Maisto produkto pridėjimas \\ \hline
	Testavimo atvejo aprašymas & Ištestuoti ar pridedant naują maisto produktą jo kiekis ir maistinės naudos specifikacija teisingai pridedama į bendrą maisto sąrašą   \\ \hline
	Testavimo eiga &1. Pagrindiniame programos lange spaudžiame mygtuką ,,ADD FOOD". 
				2. Iš pateikto sąrašo pasirenkame vieną iš produktų. 
				3. Atsidariusiame lange nurodome 100g maisto kiekio.
				4. Pasižymime kokią maisto informacija mums pateikta. 
				5. Spaudžiame mygtuką ,,Add Serving".
				6. Atsidarius pagrindiniam langui jame patikriname ar maisto informacija atitinka tą kurią mums nurodė prieš pridedant maistą.\\ \hline
	Tikėtasis rezultatas &  Pridėjus naują maisto produktą pagrindiniame lange maisto informacija atitiks tą kuri buvo parodyta pridėjimo lange\\ \hline
	Gautas Rezultatas & Maisto produktas pridėtas korektiškai  \\ \hline
	Statusas &  Atlikta\\ \hline
    \hline
    \end{tabular}
\end{center}

\begin{center}
    \begin{tabular}{ |p{5cm}|p{13cm}|}
    \hline
    	Testavimo atvejo numeris & CMT-1.1.2  \\ \hline
    	Testavimo atvejis & Maisto informacijos atvaizdavimas dideliais kiekiais  \\ \hline
	Testavimo atvejo aprašymas & Ištestuoti ar pridedant naują produktą ir nurodant jam didelius kiekius programa duomenis patetiks ir atvaizduos korektiškai.  \\ \hline
	Testavimo eiga & 1. Pagrindiniame programos lange spaudžiame mygtuką ,,ADD FOOD". 
				2. Iš pateikto sąrašo pasirenkame vieną iš produktų. 
				3. Atsidariusiame lange nurodome 2147483648g(Int32 MAX + 1) maisto kiekio.
				4. Pasižymime kokią maisto informacija mums pateikta.
				5. Spaudžiame mygtuką ,,Add Serving".
				6. Atsidarius pagrindiniam langui jame patikriname ar maisto informacija atitinka tą kurią mums nurodė prieš pridedant maistą.\\ \hline
	Tikėtasis rezultatas & Programa leis pridėti pasirinktą kiekį maisto, informacija pagrindiniame lange bus pateikta ir nurodyta korektiškai  \\ \hline
	Gautas Rezultatas & Bandant pridėti pasirinkto produkto 2147483648 gramus programa to padaryti neleidžia, kiekis automatiškai nusistato į 999999 gramus. Pagrindiniame lange pridėto produkto informacija iškraipovartotojo sąsajos elementus.  \\ \hline
	Statusas & Nepraėjo  \\ \hline
    \hline
    \end{tabular}
\end{center}

\begin{center}
    \begin{tabular}{ |p{5cm}|p{13cm}|}
    \hline
    	Testavimo atvejo numeris & CMT-1.1.3 \\ \hline
    	Testavimo atvejis & Maisto produkto pridėjimas pateikiant neskaitinę informaciją \\ \hline
	Testavimo apvejo aprašymas & Ištestuoti ar bandant pridėti pridėti maisto produktą pateikiant neskaitinį jo kiekio įvedimą(prie kiekio pateikti ne sakaičiu o raides) programa neleis pridėti produkto  \\ \hline
	Testavimo eiga & 1. Pagrindiniame programos lange spaudžiame mygtuką ,,ADD FOOD". 
				2. Iš pateikto sąrašo pasirenkame vieną iš produktų. 
				3. Atsidariusiame lange į kiekio lauką įvedame ,,test" įvestį
				4. Pasižymime kokią maisto informacija mums pateikta.
				5. Spaudžiame mygtuką ,,Add Serving".\\ \hline
	Tikėtasis rezultatas & Bandant pridėti maistą nurodant jo kiekį tekstu programa turi parodyti mums informacinį pranešimą ir neleisti mums pridėti maisto kol įvestis nebus pakeista į skaitinę įvestį \\ \hline
	Gautas Rezultatas & Bandant pridėti maistą nurodant tekstinį kiekį programa atpažysta, kad įvestis yra nekorektiška ir paryškina įvesties lauką raudonai, tačiau bandant pativirtini pridėjimą paspaudus ,,Add Serving" programa vistiek prideda maitą prie sąrašo nurodydama, kad jo pridėjome 1 gramo kiekio. \\ \hline
	Statusas & Nepraėjo \\ \hline
    \hline
    \end{tabular}
\end{center}

\begin{center}
    \begin{tabular}{ |p{5cm}|p{13cm}|}
    \hline
        Testavimo atvejo numeris & CMT-1.1.4  \\ \hline
        Testavimo atvejis & Maisto produktas nedingsta iš sąrašo perkrovus svetainę  \\ \hline
        Testavimo atvejo aprašymas & Ištestuoti ar prekrovus svetainę maisto sąrašo informacija lieka ta pati  \\ \hline
        Testavimo eiga &  1. Pagrindiniame programos lange spaudžiame mygtuką ,,ADD FOOD". 
				2. Iš pateikto sąrašo pasirenkame vieną iš produktų. 
				3. Atsidariusiame lange nurodome 100g maisto kiekio.
				4. Pasižymime kokią maisto informacija mums pateikta. 
				5. Spaudžiame mygtuką ,,Add Serving".
				6. Atsidarius pagrindiniam langui jame patikriname ar maisto informacija atitinka tą kurią mums nurodė prieš pridedant maistą.
				7. Perkrauname puslapį\\ \hline
        Tikėtasis rezultatas &  Perkrovus svetainę maisto informacija pagrindiniame saraše maisto informacija nepasikeis\\ \hline
        Gautas Rezultatas & Maisto informacija po perkrovimo pateikiama korektiškai  \\ \hline
        Statusas & Atlikta  \\ \hline
    \hline
    \end{tabular}
\end{center}

\begin{center}
    \begin{tabular}{ |p{5cm}|p{13cm}|}
    \hline
        Testavimo atvejo numeris & CMT-1.1.5  \\ \hline
        Testavimo atvejis & Maisto produktų galime pridėti tiek kiek norime  \\ \hline
        Testavimo atvejo aprašymas & Ištestuoti ar galima pridėti daugiau negu 100 skirtingų maisto vienetų į sąrašą ir pridėjus šiuos produktus informacija bus pateikta korektiškai \\ \hline
        Testavimo eiga &  1. Pagrindiniame programos lange spaudžiame mygtuką ,,ADD FOOD". 
				2. Iš pateikto sąrašo pasirenkame vieną iš produktų. 
				3. Atsidariusiame lange nurodome 100g maisto kiekio.
				4. Pasižymime kokią maisto informacija mums pateikta. 
				5. Spaudžiame mygtuką ,,Add Serving".
				6. Atsidarius pagrindiniam langui jame patikriname ar maisto informacija atitinka tą kurią mums nurodė prieš pridedant maistą.
				7. Kartojame pirmjus 6 žingsnius 101 kartą\\ \hline
        Tikėtasis rezultatas & Pridėjus daugiau negu 100 skirtingų maisto tipų informacija bus pateikta korektiškai tiek atvaizdavimo atžvilgiu tiek ir informacijos korektiškumu.\\ \hline
        Gautas Rezultatas & Pridėjus 101 maisto vienetą informacija saraše pateikta ir nurodoma korektiškai  \\ \hline
        Statusas & Atlikta  \\ \hline
    \hline
    \end{tabular}
\end{center}

\begin{center}
    \begin{tabular}{ |p{5cm}|p{13cm}|}
    \hline
        Testavimo atvejo numeris &  CMT-1.1.6\\ \hline
        Testavimo atvejis & Galime pridėti tą patį maisto produktą  \\ \hline
        Testavimo atvejo aprašymas & Ištestuoti ar pridedant tą patį maisto produktą prie sąrašo naują įrašą rodo kaip atskirą elementą saraše  \\ \hline
        Testavimo eiga &  1. Pagrindiniame programos lange spaudžiame mygtuką ,,ADD FOOD". 
				2. Iš pateikto sąrašo pasirenkame vieną iš produktų. 
				3. Atsidariusiame lange nurodome 100g maisto kiekio.
				4. Pasižymime kokią maisto informacija mums pateikta. 
				5. Spaudžiame mygtuką ,,Add Serving".
				6. Pagrindiniame programos lange spaudžiame mygtuką ,,ADD FOOD".
				7. Iš pateikto sąrašo pasirenkame tą patį produktą kurį rinkomės 2 veiksme.
				8. Atsidariusiame lange nurodome 100g maisto kiekio.
				9. Pasižymime kokią maisto informacija mums pateikta.
				10. Spaudžiame mygtuką ,,Add Serving".
				11. Pagrindiniame lange patikriname ar tas pats maistas saraše buvo pridėtas kaip atskiras elementas \\ \hline
        Tikėtasis rezultatas &  Du kartus pridėjus tą patį maistą pagrindiniame saraše bus atvaizduoti du atskiri maisto elementai\\ \hline
        Gautas Rezultatas &  Pridėjus du identiškus maisto elementus į sąrašą jie saraše buvo atvaizduoti kaip du atskiri elementai\\ \hline
        Statusas & Atlikta \\ \hline
    \hline
    \end{tabular}
\end{center}

\begin{center}
    \begin{tabular}{ |p{5cm}|p{13cm}|}
    \hline
        Testavimo atvejo numeris & CMT-1.1.7  \\ \hline
        Testavimo atvejis & Maisto kiekio keitimas pagrindiniame lange \\ \hline
        Testavimo atvejo aprašymas & Ištestuoti ar pakeitus maisto kiekį pagrindiniame lange jis korektiškai pakeičiamas ir bendra informacija pasikeičia korektiškai \\ \hline
        Testavimo eiga &  1. Pagrindiniame programos lange spaudžiame mygtuką ,,ADD FOOD". 
				2. Iš pateikto sąrašo pasirenkame vieną iš produktų. 
				3. Atsidariusiame lange nurodome 100g maisto kiekio.
				4. Pasižymime kokią maisto informacija mums pateikta. 
				5. Spaudžiame mygtuką ,,Add Serving".
				6. Pagrindiniame lange paspaudžiam ant sąrašo elemento maisto kiekio skaičiaus
				7. Pakeičiame elemento kiekį į 200 gramų
				8. Patikriname ar maisto elemento informacija pasikeitė korektiškai
				9. Patikriname ar bendra maisto informacija pasikeitė korektiškai\\ \hline
        Tikėtasis rezultatas &  Pakeitus vieno iš sąrašo elemetų kiekį jis bus korektiškai pakeistas tiek sąrašo elemento informacijoje tiek ir bendroje maisto informacijoje \\ \hline
        Gautas Rezultatas & Pakeitus vieno iš sąrašo elementų kiekį visa informacija pasikeitė korektiškai \\ \hline
        Statusas &  Atlikta\\ \hline
    \hline
    \end{tabular}
\end{center}

\begin{center}
    \begin{tabular}{ |p{5cm}|p{13cm}|}
    \hline
        Testavimo atvejo numeris & CMT-1.1.8  \\ \hline
        Testavimo atvejis & Maisto kalorijų keitimas pagrindiniame lange \\ \hline
        Testavimo atvejo aprašymas & Ištestuoti ar pakeitus maisto kalorijas pagrindiniame lange jis korektiškai pakeičiamas ir bendra informacija pasikeičia korektiškai \\ \hline
        Testavimo eiga &  1. Pagrindiniame programos lange spaudžiame mygtuką ,,ADD FOOD". 
				2. Iš pateikto sąrašo pasirenkame vieną iš produktų. 
				3. Atsidariusiame lange nurodome 100g maisto kiekio.
				4. Pasižymime kokią maisto informacija mums pateikta. 
				5. Spaudžiame mygtuką ,,Add Serving".
				6. Pagrindiniame lange paspaudžiam ant sąrašo elemento maisto kalorijų skaičiaus
				7. Pakeičiame elemento kalorijas pakeičiame į 100 kalorijų
				8. Patikriname ar maisto elemento informacija pasikeitė korektiškai
				9. Patikriname ar bendra maisto informacija pasikeitė korektiškai\\ \hline
        Tikėtasis rezultatas &  Pakeitus vieno iš sąrašo elemetų kalorijas jis bus korektiškai pakeistas tiek sąrašo elemento informacijoje tiek ir bendroje maisto informacijoje \\ \hline
        Gautas Rezultatas & Pakeitus vieno iš sąrašo elementų kalorijas visa informacija pasikeitė korektiškai \\ \hline
        Statusas &  Atlikta\\ \hline
    \hline
    \end{tabular}
\end{center}

\iffalse XXXXXXXXXXXXXXXXXXXXXXXXXXXXXXXXXXXXXXXXXXXXXXXXXXXXXXXXXXXXXXXXXXXXXXXXXXXXXXXXXXXXXXXXXXXXXXXXXXXXXXXXXXXXXXXXXXXXXXXXXXXXXXXXXXXXXXX \fi

\begin{center}
    \begin{tabular}{ |p{5cm}|p{13cm}|}
    \hline
    	Testavimo atvejo numeris & CMT-1.2.1\\ \hline
    	Testavimo atvejis & Sporto veiklos pridėjimas \\ \hline
	Testavimo atvejo aprašymas & Ištestuoti ar pridedant naują sporto veiklą jos trukmė ir sudegintos kalorijos korektiškai atsivaizduoja pagrindiniame lange    \\ \hline
	Testavimo eiga &1. Pagrindiniame programos lange spaudžiame mygtuką ,,ADD EXERCISE". 
				2. Iš pateikto sąrašo pasirenkame vieną iš sporto veiklų. 
				3. Atsidariusiame lange nurodome 30 minučių trukmę.
				4. Pasižymime kokią kalorijų informaciją mums pateikta. 
				5. Spaudžiame mygtuką ,,Add Exercise".
				6. Atsidarius pagrindiniam langui jame patikriname ar sporto veiklos informacija atitinka tą kurią mums nurodė prieš pridedant sporto veiklą.\\ \hline
	Tikėtasis rezultatas &  Pridėjus naują sporto veiklą pagrindiniame lange sportinės veiklos informacija atitiks tą kuri buvo parodyta pridėjimo lange\\ \hline
	Gautas Rezultatas & Sporto veikla pridėtas korektiškai  \\ \hline
	Statusas &  Atlikta\\ \hline
    \hline
    \end{tabular}
\end{center}

\begin{center}
    \begin{tabular}{ |p{5cm}|p{13cm}|}
    \hline
    	Testavimo atvejo numeris & CMT-1.2.2  \\ \hline
    	Testavimo atvejis & Sporto veiklos informacijos atvaizdavimas dideliais kiekiais  \\ \hline
	Testavimo atvejo aprašymas & Ištestuoti ar pridedant naują sporto veiklą ir nurodant jai didelę trukmę programa duomenis pateiks ir atvaizduos korektiškai.  \\ \hline
	Testavimo eiga & 1. Pagrindiniame programos lange spaudžiame mygtuką ,,ADD EXERCISE". 
				2. Iš pateikto sąrašo pasirenkame vieną iš sporto veiklų. 
				3. Atsidariusiame lange nurodome 2147483648(Int32 MAX + 1) minučių.
				4. Pasižymime kokią kalorijų sudeginimo informaciją mums pateikė.
				5. Spaudžiame mygtuką ,,Add Exercise".
				6. Atsidarius pagrindiniam langui jame patikriname ar sporto veiklos informacija atitinka tą kurią mums nurodė prieš pridedant sporto veiklą.\\ \hline
	Tikėtasis rezultatas & Programa leis pridėti pasirinktą sporto veiklą su ilga trukme, informacija pagrindiniame lange bus pateikta ir nurodyta korektiškai  \\ \hline
	Gautas Rezultatas & Bandant pridėti pasirinktą sporto veiklą nurodant 2147483648 minučių trukmę programa to padaryti neleidžia, trukmę automatiškai nusistato į 999999 minutes. Pagrindiniame lange pridėto produkto informacija iškraipo vartotojo sąsajos elementus.  \\ \hline
	Statusas & Nepraėjo  \\ \hline
    \hline
    \end{tabular}
\end{center}

\begin{center}
    \begin{tabular}{ |p{5cm}|p{13cm}|}
    \hline
    	Testavimo atvejo numeris & CMT-1.2.3 \\ \hline
    	Testavimo atvejis &Sporto veiklos pridėjimas pateikiant neskaitinę informaciją \\ \hline
	Testavimo apvejo aprašymas & Ištestuoti ar bandant pridėti sporto veiklą pateikiant neskaitinį jo kiekio įvedimą(prie kiekio pateikti ne sakaičiu o raides) programa neleis pridėti sporto veiklos  \\ \hline
	Testavimo eiga & 1. Pagrindiniame programos lange spaudžiame mygtuką ,,ADD EXERCISE". 
				2. Iš pateikto sąrašo pasirenkame vieną iš sporto veiklų. 
				3. Atsidariusiame lange į trukėms lauką įvedame ,,test" įvestį
				4. Pasižymime kokią sudegintų kalorijų informacija mums pateikta.
				5. Spaudžiame mygtuką ,,Add Serving".\\ \hline
	Tikėtasis rezultatas & Bandant pridėti sporto veiklą jos trukmę nurodant tekstu programa turi parodyti mums informacinį pranešimą ir neleisti mums pridėti sporto veiklos kol įvestis nebus pakeista į skaitinę įvestį \\ \hline
	Gautas Rezultatas & Bandant pridėti sporto veiklą nurodant trukme tekstu programa atpažysta, kad įvestis yra nekorektiška ir paryškina įvesties lauką raudonai, tačiau bandant pativirtini pridėjimą paspaudus ,,Add Exercise" programa vistiek prideda sporto veiklą prie sąrašo nurodydama, kad jos trukmė yra 30 minutės. \\ \hline
	Statusas & Nepraėjo \\ \hline
    \hline
    \end{tabular}
\end{center}

\begin{center}
    \begin{tabular}{ |p{5cm}|p{13cm}|}
    \hline
        Testavimo atvejo numeris & CMT-1.2.4  \\ \hline
        Testavimo atvejis & Sporto veikla nedingsta iš sąrašo perkrovus svetainę  \\ \hline
        Testavimo atvejo aprašymas & Ištestuoti ar prekrovus svetainę sporto veiklos sąrašo informacija lieka ta pati  \\ \hline
        Testavimo eiga &  1. Pagrindiniame programos lange spaudžiame mygtuką ,,ADD EXERCISE". 
				2. Iš pateikto sąrašo pasirenkame vieną iš sporto veiklų. 
				3. Atsidariusiame lange nurodome 30 minučių trukmę.
				4. Pasižymime kokią sudegintų kalorijų informacija mums pateikta. 
				5. Spaudžiame mygtuką ,,Add Exercise".
				6. Atsidarius pagrindiniam langui jame patikriname ar sporto veiklos informacija atitinka tą kurią mums nurodė prieš pridedant sporto veiklą.
				7. Perkrauname puslapį \\ \hline
        Tikėtasis rezultatas &  Perkrovus svetainę sporto veiklos informacija pagrindiniame saraše sporto veiklos informacija nepasikeis\\ \hline
        Gautas Rezultatas & Sporto veiklos informacija po perkrovimo pateikiama korektiškai  \\ \hline
        Statusas & Atlikta  \\ \hline
    \hline
    \end{tabular}
\end{center}

\begin{center}
    \begin{tabular}{ |p{5cm}|p{13cm}|}
    \hline
        Testavimo atvejo numeris & CMT-1.2.5  \\ \hline
        Testavimo atvejis & Sporto veiklų galime pridėti tiek kiek norime  \\ \hline
        Testavimo atvejo aprašymas & Ištestuoti ar galima pridėti daugiau negu 100 skirtingų sporto veikų į sąrašą ir pridėjus šiuos produktus informacija bus pateikta korektiškai \\ \hline
        Testavimo eiga &  1. Pagrindiniame programos lange spaudžiame mygtuką ,,ADD EXERCISE". 
				2. Iš pateikto sąrašo pasirenkame vieną iš sporto veiklų. 
				3. Atsidariusiame lange nurodome 30 minučių trukmę.
				4. Pasižymime koka sudegintų kalorijų informacija mums pateikta. 
				5. Spaudžiame mygtuką ,,Add Exercise".
				6. Atsidarius pagrindiniam langui jame patikriname ar sporto veiklos informacija atitinka tą kurią mums nurodė prieš pridedant sporto veiklą.
				7. Kartojame pirmjus 6 žingsnius 101 kartą \\ \hline
        Tikėtasis rezultatas & Pridėjus daugiau negu 100 skirtingų sporto veiklų tipų informacija bus pateikta korektiškai tiek atvaizdavimo atžvilgiu tiek ir informacijos korektiškumu.\\ \hline
        Gautas Rezultatas & Pridėjus 101 sporto veiklų vienetą informacija saraše pateikta ir nurodoma korektiškai  \\ \hline
        Statusas & Atlikta  \\ \hline
    \hline
    \end{tabular}
\end{center}

\begin{center}
    \begin{tabular}{ |p{5cm}|p{13cm}|}
    \hline
        Testavimo atvejo numeris &  CMT-1.2.6\\ \hline
        Testavimo atvejis & Galime pridėti tą pačią sporto veiklą  \\ \hline
        Testavimo atvejo aprašymas & Ištestuoti ar pridedant tą pačią sporto veiklą prie sąrašo naują įrašą rodo kaip atskirą elementą saraše  \\ \hline
        Testavimo eiga &  1. Pagrindiniame programos lange spaudžiame mygtuką ,,ADD EXERCISE". 
				2. Iš pateikto sąrašo pasirenkame vieną iš sporto veiklų. 
				3. Atsidariusiame lange nurodome 30 minučių trukmę.
				4. Pasižymime kokią sudegintų kalorijų informacija mums pateikta. 
				5. Spaudžiame mygtuką ,,Add Exercise".
				6. Pagrindiniame programos lange spaudžiame mygtuką ,,ADD EXERCISE".
				7. Iš pateikto sąrašo pasirenkame tą pačią sporto veiklą kurią rinkomės 2 veiksme.
				8. Atsidariusiame lange nurodome 30 minučių trukmę.
				9. Pasižymime kokią sudegintų kalorijų informacija mums pateikta.
				10. Spaudžiame mygtuką ,,Add Exercise".
				11. Pagrindiniame lange patikriname ar ta pati ssporto veikla saraše buvo pridėtas kaip atskiras elementas \\ \hline
        Tikėtasis rezultatas &  Du kartus pridėjus tą pačią sporto veiklą pagrindiniame saraše bus atvaizduoti du atskiri sporto veiklos elementai\\ \hline
        Gautas Rezultatas &  Pridėjus du identiškus sporto veiklos elementus į sąrašą jie saraše buvo atvaizduoti kaip du atskiri elementai korektiškai\\ \hline
        Statusas & Atlikta \\ \hline
    \hline
    \end{tabular}
\end{center}

\begin{center}
    \begin{tabular}{ |p{5cm}|p{13cm}|}
    \hline
        Testavimo atvejo numeris & CMT-1.2.7  \\ \hline
        Testavimo atvejis & Sporto veiklos trukmės keitimas pagrindiniame lange \\ \hline
        Testavimo atvejo aprašymas & Ištestuoti ar pakeitus sporto veiklos trukmę pagrindiniame lange jis korektiškai pakeičiamas ir bendra informacija pasikeičia korektiškai \\ \hline
        Testavimo eiga &  1. Pagrindiniame programos lange spaudžiame mygtuką ,,ADD EXERCISE". 
				2. Iš pateikto sąrašo pasirenkame vieną iš sporto veiklų. 
				3. Atsidariusiame lange nurodome 30 minučių trukmę.
				4. Pasižymime kokią sudegintų kalorijų informacija mums pateikta. 
				5. Spaudžiame mygtuką ,,Add Exercise".
				6. Pagrindiniame lange paspaudžiam ant sąrašo elemento sporto veiklos trukmės skaičiaus
				7. Pakeičiame elemento trukmę į 60 minučių
				8. Patikriname ar sporto veiklos elemento informacija pasikeitė korektiškai
				9. Patikriname ar bendra sudegintų kalorijų informacija pasikeitė korektiškai\\ \hline
        Tikėtasis rezultatas &  Pakeitus vieno iš sąrašo elemetų trukmę ji bus korektiškai pakeistas tiek sąrašo elemento informacijoje tiek ir bendroje sudegintų kalorijų informacijoje \\ \hline
        Gautas Rezultatas & Pakeitus vieno iš sąrašo elementų trukmę visa informacija pasikeitė korektiškai \\ \hline
        Statusas &  Atlikta\\ \hline
    \hline
    \end{tabular}
\end{center}

\iffalse XXXXXXXXXXXXXXXXXXXXXXXXXXXXXXXXXXXXXXXXXXXXXXXXXXXXXXXXXXXXXXXXXXXXXXXXXXXXXXXXXXXXXXXXXXXXXXXXXXXXXXXXXXXXXXXXXXXXXXXXXXXXXXXXXXXXXXX \fi
\begin{center}
    \begin{tabular}{ |p{5cm}|p{13cm}|}
    \hline
    	Testavimo atvejo numeris & CMT-1.3.1\\ \hline
    	Testavimo atvejis & Biometrikos pridėjimas \\ \hline
	Testavimo atvejo aprašymas & Ištestuoti ar pridedant naują biometriką jos reikšmė teisingai pridedama į bendrą sąrašą   \\ \hline
	Testavimo eiga &1. Pagrindiniame programos lange spaudžiame mygtuką ,,ADD BIOMETRIC". 
				2. Iš pateikto sąrašo pasirenkame vieną iš biometrikų. 
				3. Atsidariusiame lange nurodome 100 kaip biometrikos reikšmę.
				4. Spaudžiame mygtuką ,,Add Measurement".
				5. Atsidarius pagrindiniam langui jame patikriname ar biometrinė informacija atitinka tą kurią nurodėme.\\ \hline
	Tikėtasis rezultatas &  Pridėjus naują biometriką pagrindiniame lange biometrikos informacija atitiks tą kuri buvome nurodę\\ \hline
	Gautas Rezultatas & Biometrika pridėta korektiškai  \\ \hline
	Statusas &  Atlikta\\ \hline
    \hline
    \end{tabular}
\end{center}

\begin{center}
    \begin{tabular}{ |p{5cm}|p{13cm}|}
    \hline
    	Testavimo atvejo numeris & CMT-1.3.2  \\ \hline
    	Testavimo atvejis & Biometrikos atvaizdavimas nurodant didelias reikšmes  \\ \hline
	Testavimo atvejo aprašymas & Ištestuoti ar pridedant naują biometriką ir nurodant jai didelę reikšmę programa duomenis pateiks ir atvaizduos korektiškai.  \\ \hline
	Testavimo eiga & 1. Pagrindiniame programos lange spaudžiame mygtuką ,,ADD BIOMETRIC". 
				2. Iš pateikto sąrašo pasirenkame vieną iš biometrikų. 
				3. Atsidariusiame lange nurodome 2147483648(Int32 MAX + 1) reikšmę.
				4. Spaudžiame mygtuką ,,Add Measurement".
				5. Atsidarius pagrindiniam langui jame patikriname ar biometrikos informacija atitinka tą kurią nurodėme.\\ \hline
	Tikėtasis rezultatas & Programa leis pridėti pasirinktą sporto veiklą su ilga trukme, informacija pagrindiniame lange bus pateikta ir nurodyta korektiškai  \\ \hline
	Gautas Rezultatas & Bandant pridėti biometriką nurodant 2147483648 reikšmę programa to padaryti neleidžia, reikšmė automatiškai nusistato į 999999 minutes. Pagrindiniame lange pridėtos biometrikos informacija iškraipo vartotojo sąsajos elementus.  \\ \hline
	Statusas & Nepraėjo  \\ \hline
    \hline
    \end{tabular}
\end{center}

\begin{center}
    \begin{tabular}{ |p{5cm}|p{13cm}|}
    \hline
    	Testavimo atvejo numeris & CMT-1.3.3 \\ \hline
    	Testavimo atvejis &Biometrikos pridėjimas pateikiant neskaitinę informaciją \\ \hline
	Testavimo apvejo aprašymas & Ištestuoti ar bandant pridėti biometriką pateikiant neskaitinę jos reikšmę įvedimą(prie reikšmės pateikti ne sakaičiu o raides) programa neleis pridėti biometrikos  \\ \hline
	Testavimo eiga & 1. Pagrindiniame programos lange spaudžiame mygtuką ,,ADD BIOMETRIC". 
				2. Iš pateikto sąrašo pasirenkame vieną iš biometrikų. 
				3. Atsidariusiame lange į reikšmės lauką įvedame ,,test" įvestį
				4. Spaudžiame mygtuką ,,Add Measurement".\\ \hline
	Tikėtasis rezultatas & Bandant pridėti biometriką jos reikšmę nurodant tekstu programa turi parodyti mums informacinį pranešimą ir neleisti mums pridėti biometrikos kol įvestis nebus pakeista į skaitinę įvestį \\ \hline
	Gautas Rezultatas & Bandant pridėti biometriką su neteisinga įvestimi programa neleidžia jos pridėti kol įvestis nebus pakeista į skaičių \\ \hline
	Statusas & Atlikta \\ \hline
    \hline
    \end{tabular}
\end{center}

\begin{center}
    \begin{tabular}{ |p{5cm}|p{13cm}|}
    \hline
        Testavimo atvejo numeris & CMT-1.3.4  \\ \hline
        Testavimo atvejis & Biometrika nedingsta iš sąrašo perkrovus svetainę  \\ \hline
        Testavimo atvejo aprašymas & Ištestuoti ar prekrovus svetainę biometrikos sąrašo informacija lieka ta pati  \\ \hline
        Testavimo eiga &  1. Pagrindiniame programos lange spaudžiame mygtuką ,,ADD BIOMETRIC". 
				2. Iš pateikto sąrašo pasirenkame vieną iš biometrikų. 
				3. Atsidariusiame lange nurodome 100 reikšmę.
				4. Spaudžiame mygtuką ,,Add Measurement".
				5. Atsidarius pagrindiniam langui jame patikriname ar biometrikos informacija atitinka tą kurią mes nurodėme
				6. Perkrauname puslapį\\ \hline
        Tikėtasis rezultatas &  Perkrovus svetainę biometrikos informacija pagrindiniame saraše biometrikos informacija nepasikeis\\ \hline
        Gautas Rezultatas & Biometrikos informacija po perkrovimo pateikiama korektiškai  \\ \hline
        Statusas & Atlikta  \\ \hline
    \hline
    \end{tabular}
\end{center}

\begin{center}
    \begin{tabular}{ |p{5cm}|p{13cm}|}
    \hline
        Testavimo atvejo numeris & CMT-1.3.5  \\ \hline
        Testavimo atvejis & Biometrikų galime pridėti tiek kiek norime  \\ \hline
        Testavimo atvejo aprašymas & Ištestuoti ar galima pridėti daugiau negu 100 skirtingų biometrikų į sąrašą ir pridėjus šias biometrikas informacija bus pateikta korektiškai \\ \hline
        Testavimo eiga &  1. Pagrindiniame programos lange spaudžiame mygtuką ,,ADD BIOMETRIC". 
				2. Iš pateikto sąrašo pasirenkame vieną iš biometrikų. 
				3. Atsidariusiame lange nurodome 100 biometrikos reikšmę.
				4. Spaudžiame mygtuką ,,Add Measurement".
				5. Atsidarius pagrindiniam langui jame patikriname ar biometrikos informacija atitinka tą kurią mes nurodėme.
				6. Kartojame pirmjus 6 žingsnius 101 kartą\\ \hline
        Tikėtasis rezultatas & Pridėjus daugiau negu 100 skirtingų biometrikos tipų informacija bus pateikta korektiškai tiek atvaizdavimo atžvilgiu tiek ir informacijos korektiškumu.\\ \hline
        Gautas Rezultatas & Pridėjus 101 biometrikos vienetą informacija saraše pateikta ir nurodoma korektiškai  \\ \hline
        Statusas & Atlikta  \\ \hline
    \hline
    \end{tabular}
\end{center}

\begin{center}
    \begin{tabular}{ |p{5cm}|p{13cm}|}
    \hline
        Testavimo atvejo numeris &  CMT-1.3.6\\ \hline
        Testavimo atvejis & Galime pridėti tą pačią biometriką \\ \hline
        Testavimo atvejo aprašymas & Ištestuoti ar pridedant tą pačią biometriką prie sąrašo naują įrašą rodo kaip atskirą elementą saraše  \\ \hline
        Testavimo eiga &  1. Pagrindiniame programos lange spaudžiame mygtuką ,,ADD BIOMETRIC". 
				2. Iš pateikto sąrašo pasirenkame vieną iš biometrikų. 
				3. Atsidariusiame lange nurodome 100 biometrikos reikšmę.
				4. Spaudžiame mygtuką ,,Add Measurement".
				5. Pagrindiniame programos lange spaudžiame mygtuką ,,ADD BIOMETRIC".
				6. Iš pateikto sąrašo pasirenkame tą pačią biometriką kurią rinkomės 2 veiksme.
				7. Atsidariusiame lange nurodome 200 biometrikos reikšmę.
				8. Spaudžiame mygtuką ,,Add Measurement".
				9. Pagrindiniame lange patikriname ar ta pati biometriką saraše buvo pridėtas kaip atskiras elementas \\ \hline
        Tikėtasis rezultatas &  Du kartus pridėjus tą pačią biometriką pagrindiniame saraše bus atvaizduoti du atskiri biometrikos elementai\\ \hline
        Gautas Rezultatas &  Pridėjus du identiškus biometrikos elementus į sąrašą jie saraše buvo atvaizduoti kaip du atskiri elementai korektiškai\\ \hline
        Statusas & Atlikta \\ \hline
    \hline
    \end{tabular}
\end{center}

\begin{center}
    \begin{tabular}{ |p{5cm}|p{13cm}|}
    \hline
        Testavimo atvejo numeris & CMT-1.3.7  \\ \hline
        Testavimo atvejis & Biometrikos reikšmės keitimas pagrindiniame lange \\ \hline
        Testavimo atvejo aprašymas & Ištestuoti ar pakeitusbiometrikos reikšmę pagrindiniame lange ji korektiškai pakeičiamas ir bendra informacija pasikeičia korektiškai \\ \hline
        Testavimo eiga &  1. Pagrindiniame programos lange spaudžiame mygtuką ,,ADD BIOMETRIC". 
				2. Iš pateikto sąrašo pasirenkame vieną iš biometrikų. 
				3. Atsidariusiame lange nurodome 100 biometrikos reikšmę.
				4. Spaudžiame mygtuką ,,Add Measurement".
				5. Pagrindiniame lange paspaudžiam ant sąrašo elemento biometrikos reikšmės skaičiaus
				6. Pakeičiame elemento reikšmę į 200
				7. Patikriname ar biometrikos elemento informacija pasikeitė korektiškai \\ \hline
        Tikėtasis rezultatas &  Pakeitus vieno iš sąrašo elemetų reikšmę ji bus korektiškai pakeistas tiek sąrašo elemento informacijoje tiek ir bendroje biometrikos informacijoje \\ \hline
        Gautas Rezultatas & Pakeitus vieno iš sąrašo elementų reikšmę visa informacija pasikeitė korektiškai \\ \hline
        Statusas &  Atlikta\\ \hline
    \hline
    \end{tabular}
\end{center}

\iffalse XXXXXXXXXXXXXXXXXXXXXXXXXXXXXXXXXXXXXXXXXXXXXXXXXXXXXXXXXXXXXXXXXXXXXXXXXXXXXXXXXXXXXXXXXXXXXXXXXXXXXXXXXXXXXXXXXXXXXXXXXXXXXXXXXXXXXXX \fi

\begin{center}
    \begin{tabular}{ |p{5cm}|p{13cm}|}
    \hline
    	Testavimo atvejo numeris & CMT-2.1.1\\ \hline
    	Testavimo atvejis & Maisto produkto pridėjimas \\ \hline
	Testavimo atvejo aprašymas & Ištestuoti ar pridedant naują maisto produktą jo kiekis ir maistinės naudos specifikacija teisingai pridedama į bendrą maisto sąrašą   \\ \hline
	Testavimo eiga &1. Pagrindiniame programos lange spaudžiame mygtuką pliuso(+) mygtuką. 
				2. Atsidaro pasirinkimu meniu kuriame renkamės ,,ADD FOOD"
				3. Atsidaro sąrašas iš kurio pasirenkame vieną iš maisto pasirinkimų
				4. Atsidaro langas kuriame nurodome maisto kiekį 100g
				5. Pasižymime pateiktą maistinę informaciją
				5. Spaudžiame ,,ADD SERVING"  mygtuką
				6. Pagrindiniame lange patikriname ar maistas buvo korektiškai pridėtas į sąrašą ir ar bendra maisto informacija atitinkamai pasikeitė \\ \hline
	Tikėtasis rezultatas &  Pridėjus naują maisto produktą pagrindiniame lange maisto informacija atitiks tą kuri buvo parodyta pridėjimo lange\\ \hline
	Gautas Rezultatas & Maisto produktas pridėtas korektiškai  \\ \hline
	Statusas &  Atlikta\\ \hline
    \hline
    \end{tabular}
\end{center}

\begin{center}
    \begin{tabular}{ |p{5cm}|p{13cm}|}
    \hline
    	Testavimo atvejo numeris & CMT-2.1.2  \\ \hline
    	Testavimo atvejis & Maisto informacijos atvaizdavimas dideliais kiekiais  \\ \hline
	Testavimo atvejo aprašymas & Ištestuoti ar pridedant naują produktą ir nurodant jam didelius kiekius programa duomenis pateiks ir atvaizduos korektiškai.  \\ \hline
	Testavimo eiga & 1. Pagrindiniame programos lange spaudžiame mygtuką pliuso(+) mygtuką. 
				2. Atsidaro pasirinkimu meniu kuriame renkamės ,,ADD FOOD"
				3. Atsidaro sąrašas iš kurio pasirenkame vieną iš maisto pasirinkimų
				4. Atsidaro langas kuriame nurodome maisto kiekį 2147483648(Int32 MAX + 1) gramus.
				5. Pasižymime pateiktą maistinę informaciją
				6. Spaudžiame ,,ADD SERVING"  mygtuką
				7. Pagrindiniame lange patikriname ar maistas buvo korektiškai pridėtas į sąrašą ir ar bendra maisto informacija atitinkamai pasikeitė\\ \hline
	Tikėtasis rezultatas & Programa leis pridėti pasirinktą kiekį maisto, informacija pagrindiniame lange bus pateikta ir nurodyta korektiškai  \\ \hline
	Gautas Rezultatas & Net ir įvedant labai didelius skaičius programa informacija atvaizduoja korektiškai ir vartotojo sąsaja nebūna iškraipoma. Skaičiai būna atvaizduojami moksliniu formatu(1.2E16) \\ \hline
	Statusas & Atlikta  \\ \hline
    \hline
    \end{tabular}
\end{center}

\begin{center}
    \begin{tabular}{ |p{5cm}|p{13cm}|}
    \hline
    	Testavimo atvejo numeris & CMT-2.1.3 \\ \hline
    	Testavimo atvejis & Maisto produkto pridėjimas pateikiant neskaitinę informaciją \\ \hline
	Testavimo apvejo aprašymas & Ištestuoti ar bandant pridėti pridėti maisto produktą pateikiant neskaitinį jo kiekio įvedimą(prie kiekio pateikti ne sakaičiu o raides) aplikacija neleis pridėti produkto  \\ \hline
	Testavimo eiga & 1. Pagrindiniame programos lange spaudžiame mygtuką pliuso(+) mygtuką. 
				2. Atsidaro pasirinkimu meniu kuriame renkamės ,,ADD FOOD"
				3. Atsidaro sąrašas iš kurio pasirenkame vieną iš maisto pasirinkimų
				4. Atsidaro langas kuriame nurodome maisto kiekį kaip ,,test"
				5. Aplikacija neleidžia pridėti maisto kol įvesties nepakeičiame į skaitine įvestimi \\ \hline
	Tikėtasis rezultatas & Bandant pridėti maistą nurodant jo kiekį tekstu aplikacija turi parodyti mums informacinį pranešimą ir neleisti mums pridėti maisto kol įvestis nebus pakeista į skaitinę įvestį \\ \hline
	Gautas Rezultatas & Pridedant maistą aplikacija neleidžia įvesti teksto, vienintelė galimybė yra vesti skaičius \\ \hline
	Statusas & Atlikta \\ \hline
    \hline
    \end{tabular}
\end{center}

\begin{center}
    \begin{tabular}{ |p{5cm}|p{13cm}|}
    \hline
        Testavimo atvejo numeris & CMT-2.1.4  \\ \hline
        Testavimo atvejis & Maisto produktas nedingsta iš sąrašo įšjungus ir vėl įjungus aplikaciją  \\ \hline
        Testavimo atvejo aprašymas & Ištestuoti ar išjungus ir vėl įjungus aplikaciją maisto sąrašo informacija lieka ta pati  \\ \hline
        Testavimo eiga &  1. Pagrindiniame programos lange spaudžiame mygtuką pliuso(+) mygtuką. 
				2. Atsidaro pasirinkimu meniu kuriame renkamės ,,ADD FOOD"
				3. Atsidaro sąrašas iš kurio pasirenkame vieną iš maisto pasirinkimų
				4. Atsidaro langas kuriame nurodome maisto kiekį 100g
				5. Pasižymime pateiktą maistinę informaciją
				5. Spaudžiame ,,ADD SERVING"  mygtuką
				6. Pagrindiniame lange patikriname ar maistas buvo korektiškai pridėtas į sąrašą ir ar bendra maisto informacija atitinkamai pasikeitė 
				7. Išjungiame aplikaciją
				8. Ijungiame aplikaciją
				9. Pagrindiniame lange patikriname ar infoirmacija apie maisto informacija išliko tokia pati\\ \hline
        Tikėtasis rezultatas &  Išjungus ir vėl įjungus aplikaciją maisto informacija pagrindiniame saraše maisto informacija nepasikeis\\ \hline
        Gautas Rezultatas & Maisto informacija po išjungimo ir įjungimo pateikiama korektiškai  \\ \hline
        Statusas & Atlikta  \\ \hline
    \hline
    \end{tabular}
\end{center}

\begin{center}
    \begin{tabular}{ |p{5cm}|p{13cm}|}
    \hline
        Testavimo atvejo numeris & CMT-2.1.5  \\ \hline
        Testavimo atvejis & Maisto produktų galime pridėti tiek kiek norime  \\ \hline
        Testavimo atvejo aprašymas & Ištestuoti ar galima pridėti daugiau negu 100 skirtingų maisto vienetų į sąrašą ir pridėjus šiuos produktus informacija bus pateikta korektiškai \\ \hline
        Testavimo eiga & 1. Pagrindiniame programos lange spaudžiame mygtuką pliuso(+) mygtuką. 
				2. Atsidaro pasirinkimu meniu kuriame renkamės ,,ADD FOOD"
				3. Atsidaro sąrašas iš kurio pasirenkame vieną iš maisto pasirinkimų
				4. Atsidaro langas kuriame nurodome maisto kiekį 100g
				5. Pasižymime pateiktą maistinę informaciją
				5. Spaudžiame ,,ADD SERVING"  mygtuką
				6. Pagrindiniame lange patikriname ar maistas buvo korektiškai pridėtas į sąrašą ir ar bendra maisto informacija atitinkamai pasikeitė 
				7. Kartojame 1 - 6 žingsnius 101 kartą
				8. Patikriname ar pagrindiniame ekrane visa informacija atvaizduota korektiškai \\ \hline
        Tikėtasis rezultatas & Pridėjus daugiau negu 100 skirtingų maisto tipų informacija bus pateikta korektiškai tiek atvaizdavimo atžvilgiu tiek ir informacijos korektiškumu.\\ \hline
        Gautas Rezultatas & Pridėjus 101 maisto vienetą informacija saraše pateikta ir nurodoma korektiškai  \\ \hline
        Statusas & Atlikta  \\ \hline
    \hline
    \end{tabular}
\end{center}

\begin{center}
    \begin{tabular}{ |p{5cm}|p{13cm}|}
    \hline
        Testavimo atvejo numeris &  CMT-2.1.6\\ \hline
        Testavimo atvejis & Galime pridėti tą patį maisto produktą  \\ \hline
        Testavimo atvejo aprašymas & Ištestuoti ar pridedant tą patį maisto produktą prie sąrašo naują įrašą rodo kaip atskirą elementą saraše  \\ \hline
        Testavimo eiga &  1. Pagrindiniame aplikacijos lange spaudžiame mygtuką pliuso(+) mygtuką. 
				2. Atsidaro pasirinkimu meniu kuriame renkamės ,,ADD FOOD"
				3. Atsidaro sąrašas iš kurio pasirenkame vieną iš maisto pasirinkimų
				4. Atsidaro langas kuriame nurodome maisto kiekį 100g
				5. Spaudžiame ,,ADD SERVING"  mygtuką
				6. Pagrindiniame lange patikriname ar maistas buvo korektiškai pridėtas į sąrašą ir ar bendra maisto informacija atitinkamai pasikeitė
				7. Pakartojame 1 - 6 žingsnius
				8. Patikriname ar pagrindiniame aplikacijos ekrane buvo pridėti du maisto elementai \\ \hline
        Tikėtasis rezultatas &  Du kartus pridėjus tą patį maistą pagrindiniame saraše bus atvaizduoti du atskiri maisto elementai\\ \hline
        Gautas Rezultatas &  Pridėjus du identiškus maisto elementus į sąrašą jie saraše buvo atvaizduoti kaip du atskiri elementai\\ \hline
        Statusas & Atlikta \\ \hline
    \hline
    \end{tabular}
\end{center}

\begin{center}
    \begin{tabular}{ |p{5cm}|p{13cm}|}
    \hline
        Testavimo atvejo numeris & CMT-2.1.7  \\ \hline
        Testavimo atvejis & Maisto kiekio keitimas pagrindiniame lange \\ \hline
        Testavimo atvejo aprašymas & Ištestuoti ar pakeitus maisto kiekį pagrindiniame lange jis korektiškai pakeičiamas ir bendra informacija pasikeičia korektiškai \\ \hline
        Testavimo eiga &  1. Pagrindiniame aplikacijos lange spaudžiame mygtuką pliuso(+) mygtuką. 
				2. Atsidaro pasirinkimu meniu kuriame renkamės ,,ADD FOOD"
				3. Atsidaro sąrašas iš kurio pasirenkame vieną iš maisto pasirinkimų
				4. Atsidaro langas kuriame nurodome maisto kiekį 100g
				5. Spaudžiame ,,ADD SERVING"  mygtuką
				6. Pagrindiniame lange patikriname ar maistas buvo korektiškai pridėtas į sąrašą ir ar bendra maisto informacija atitinkamai pasikeitė
				7. Pagrindiniame lange pasirenkame maisto elementą
				8. Atsidaro langas kuriame galime pakeisti maisto kiekį
				9. Pakeičiame maisto kiekį į 200 gramų
				10. Pagrindiniame lange patikriname ar maisto kiekis buvo pakeistas korektiškai \\ \hline
        Tikėtasis rezultatas &  Pakeitus vieno iš sąrašo elemetų kiekį jis bus korektiškai pakeistas tiek sąrašo elemento informacijoje tiek ir bendroje maisto informacijoje \\ \hline
        Gautas Rezultatas & Pakeitus vieno iš sąrašo elementų kiekį visa informacija pasikeitė korektiškai \\ \hline
        Statusas &  Atlikta\\ \hline
    \hline
    \end{tabular}
\end{center}


\iffalse XXXXXXXXXXXXXXXXXXXXXXXXXXXXXXXXXXXXXXXXXXXXXXXXXXXXXXXXXXXXXXXXXXXXXXXXXXXXXXXXXXXXXXXXXXXXXXXXXXXXXXXXXXXXXXXXXXXXXXXXXXXXXXXXXXXXXXX \fi

\begin{center}
    \begin{tabular}{ |p{5cm}|p{13cm}|}
    \hline
    	Testavimo atvejo numeris & CMT-2.2.1\\ \hline
    	Testavimo atvejis & Sporto veiklos pridėjimas \\ \hline
	Testavimo atvejo aprašymas & Ištestuoti ar pridedant naują sporto veiklą jos trukmė ir sudegintos kalorijos korektiškai atsivaizduoja pagrindiniame lange    \\ \hline
	Testavimo eiga &1. Pagrindiniame aplikacijos lange spaudžiame mygtuką pliuso(+) mygtuką. 
				2. Atsidaro pasirinkimu meniu kuriame renkamės ,,ADD ACTIVITY"
				3. Atsidaro sąrašas iš kurio pasirenkame vieną iš sporto veiklų
				4. Atsidaro langas kuriame nurodome sportavimo laiką
				5. Mums parodoma sudegintų kalorijų skaičius per pasirinktą laiką
				5. Spaudžiame ,,ADD EXERCISE"  mygtuką
				6. Pagrindiniame lange patikriname sporto veikla ir sudegintos kalorijos buvo atvaizduotos teisingai\\ \hline
	Tikėtasis rezultatas &  Pridėjus naują sporto veiklą pagrindiniame lange sportinės veiklos informacija atitiks tą kuri buvo parodyta pridėjimo lange\\ \hline
	Gautas Rezultatas & Sporto veikla pridėtas korektiškai  \\ \hline
	Statusas &  Atlikta\\ \hline
    \hline
    \end{tabular}
\end{center}

\begin{center}
    \begin{tabular}{ |p{5cm}|p{13cm}|}
    \hline
    	Testavimo atvejo numeris & CMT-2.2.2  \\ \hline
    	Testavimo atvejis & Sporto veiklos informacijos atvaizdavimas dideliais kiekiais  \\ \hline
	Testavimo atvejo aprašymas & Ištestuoti ar pridedant naują sporto veiklą ir nurodant jai didelę trukmę aplikacija duomenis pateiks ir atvaizduos korektiškai.  \\ \hline
	Testavimo eiga & 1. Pagrindiniame aplikacijos lange spaudžiame mygtuką pliuso(+) mygtuką. 
				2. Atsidaro pasirinkimu meniu kuriame renkamės ,,ADD ACTIVITY"
				3. Atsidaro sąrašas iš kurio pasirenkame vieną iš sporto veiklų
				4. Atsidaro langas kuriame nurodome sportavimo laiką  2147483648(INT32 MAX + 1) minučių
				5. Mums parodoma sudegintų kalorijų skaičius per pasirinktą laiką
				5. Spaudžiame ,,ADD EXERCISE"  mygtuką
				6. Pagrindiniame lange patikriname sporto veikla ir sudegintos kalorijos buvo atvaizduotos teisingai ir didelis skaičius neįtakojo vartotojo sąsajos\\ \hline
	Tikėtasis rezultatas & Programa leis pridėti pasirinktą sporto veiklą su ilga trukme, informacija pagrindiniame lange bus pateikta ir nurodyta korektiškai  \\ \hline
	Gautas Rezultatas & Bandant pridėti pasirinktą sporto veiklą nurodant 2147483648 minučių trukmę aplikaciją duomenis pateikia korektiškai ir vartotojo sąsaja nebūna įtakojama. Su dideliais skaičiais aplikacija naudoja mokslinę notaciją 96E16.  \\ \hline
	Statusas & Atlikta  \\ \hline
    \hline
    \end{tabular}
\end{center}

\begin{center}
    \begin{tabular}{ |p{5cm}|p{13cm}|}
    \hline
    	Testavimo atvejo numeris & CMT-2.2.3 \\ \hline
    	Testavimo atvejis &Sporto veiklos pridėjimas pateikiant neskaitinę informaciją \\ \hline
	Testavimo apvejo aprašymas & Ištestuoti ar bandant pridėti sporto veiklą pateikiant neskaitinį jo kiekio įvedimą(prie kiekio pateikti ne sakaičiu o raides) programa neleis pridėti sporto veiklos  \\ \hline
	Testavimo eiga & 1. Pagrindiniame aplikacijos lange spaudžiame mygtuką pliuso(+) mygtuką. 
				2. Atsidaro pasirinkimu meniu kuriame renkamės ,,ADD ACTIVITY"
				3. Atsidaro sąrašas iš kurio pasirenkame vieną iš sporto veiklų
				4. Atsidaro langas kuriame nurodome sportavimo laiką nurodome kaip ,,test" \\ \hline
	Tikėtasis rezultatas & Bandant pridėti sporto veiklą jos trukmę nurodant tekstu programa turi parodyti mums informacinį pranešimą ir neleisti mums pridėti sporto veiklos kol įvestis nebus pakeista į skaitinę įvestį \\ \hline
	Gautas Rezultatas & Aplikacija neleidžia sporto laiko pateikti raidėmi, nes vartotojui atidaroma tik skaičių klaviatūra \\ \hline
	Statusas & Atlikta \\ \hline
    \hline
    \end{tabular}
\end{center}

\begin{center}
    \begin{tabular}{ |p{5cm}|p{13cm}|}
    \hline
        Testavimo atvejo numeris & CMT-2.2.4  \\ \hline
        Testavimo atvejis & Sporto veikla nedingsta iš sąrašo perkrovus svetainę  \\ \hline
        Testavimo atvejo aprašymas & Ištestuoti ar prekrovus svetainę sporto veiklos sąrašo informacija lieka ta pati  \\ \hline
        Testavimo eiga &  1. Pagrindiniame aplikacijos lange spaudžiame mygtuką pliuso(+) mygtuką. 
				2. Atsidariusiame aplikacijos lange spaudžiame mygtuką ,,ADD ACTIVITY". 
				3. Iš pateikto sąrašo pasirenkame vieną iš sporto veiklų. 
				4. Atsidariusiame lange nurodome 30 minučių trukmę.
				5. Pasižymime kokią sudegintų kalorijų informacija mums pateikta. 
				6. Spaudžiame mygtuką ,,ADD EXERCISE".
				7. Atsidarius pagrindiniam langui jame patikriname ar sporto veiklos informacija atitinka tą kurią mums nurodė prieš pridedant sporto veiklą.
				8. Perkrauname puslapį\\ \hline
        Tikėtasis rezultatas &  Perkrovus svetainę sporto veiklos informacija pagrindiniame saraše sporto veiklos informacija nepasikeis\\ \hline
        Gautas Rezultatas & Sporto veiklos informacija po perkrovimo pateikiama korektiškai  \\ \hline
        Statusas & Atlikta  \\ \hline
    \hline
    \end{tabular}
\end{center}


\begin{center}
    \begin{tabular}{ |p{5cm}|p{13cm}|}
    \hline
        Testavimo atvejo numeris & CMT-2.2.5  \\ \hline
        Testavimo atvejis & Sporto veiklų galime pridėti tiek kiek norime  \\ \hline
        Testavimo atvejo aprašymas & Ištestuoti ar galima pridėti daugiau negu 100 skirtingų sporto veikų į sąrašą ir pridėjus šias veiklas informacija bus pateikta korektiškai \\ \hline
        Testavimo eiga &  1. Pagrindiniame aplikacijos lange spaudžiame mygtuką pliuso(+) mygtuką. 
				2.Atsidariusiame aplikacijos lange spaudžiame mygtuką ,,ADD ACTIVITY". 
				3. Iš pateikto sąrašo pasirenkame vieną iš sporto veiklų. 
				4. Atsidariusiame lange nurodome 30 minučių trukmę.
				5. Pasižymime kokią sudegintų kalorijų informacija mums pateikta. 
				6. Spaudžiame mygtuką ,,ADD EXERCISE".
				7. Patikriname ar informacija išsisaugojo pagrindiniame ekrane
				8. Kartoti 1 - 7 žingsnius 101 kartų \\ \hline
        Tikėtasis rezultatas & Pridėjus daugiau negu 100 skirtingų sporto veiklų tipų informacija bus pateikta korektiškai tiek atvaizdavimo atžvilgiu tiek ir informacijos korektiškumu.\\ \hline
        Gautas Rezultatas & Pridėjus 101 sporto veiklų vienetą informacija saraše pateikta ir nurodoma korektiškai  \\ \hline
        Statusas & Atlikta  \\ \hline
    \hline
    \end{tabular}
\end{center}

\begin{center}
    \begin{tabular}{ |p{5cm}|p{13cm}|}
    \hline
        Testavimo atvejo numeris &  CMT-2.2.6\\ \hline
        Testavimo atvejis & Galime pridėti tą pačią sporto veiklą  \\ \hline
        Testavimo atvejo aprašymas & Ištestuoti ar pridedant tą pačią sporto veiklą prie sąrašo naują įrašą rodo kaip atskirą elementą saraše  \\ \hline
        Testavimo eiga &  1. Pagrindiniame aplikacijos lange spaudžiame mygtuką pliuso(+) mygtuką. 
				2. Atsidariusiame aplikacijos lange spaudžiame mygtuką ,,ADD EXERCISE". 
				3. Iš pateikto sąrašo pasirenkame vieną iš sporto veiklų. 
				4. Atsidariusiame lange nurodome 30 minučių trukmę.
				5. Pasižymime kokią sudegintų kalorijų informacija mums pateikta. 
				6. Spaudžiame mygtuką ,,Add Exercise".
				7. Patikriname ar informacija išsisaugojo pagrindiniame ekrane 
				8. Pakartojame 1 - 6 žingsnius pridedant identišką sporto veiklą
				9. Pagrindiniame ekrane patikriname ar antra veikla pridėta kaip atskiras sąrašo elementas \\ \hline
        Tikėtasis rezultatas &  Du kartus pridėjus tą pačią sporto veiklą pagrindiniame saraše bus atvaizduoti du atskiri sporto veiklos elementai\\ \hline
        Gautas Rezultatas &  Pridėjus du identiškus sporto veiklos elementus į sąrašą jie saraše buvo atvaizduoti kaip du atskiri elementai korektiškai\\ \hline
        Statusas & Atlikta \\ \hline
    \hline
    \end{tabular}
\end{center}

\begin{center}
    \begin{tabular}{ |p{5cm}|p{13cm}|}
    \hline
        Testavimo atvejo numeris & CMT-2.2.7  \\ \hline
        Testavimo atvejis & Sporto veiklos trukmės keitimas pagrindiniame lange \\ \hline
        Testavimo atvejo aprašymas & Ištestuoti ar pakeitus sporto veiklos trukmę pagrindiniame lange jis korektiškai pakeičiamas ir bendra informacija pasikeičia korektiškai \\ \hline
        Testavimo eiga &   1. Pagrindiniame aplikacijos lange spaudžiame mygtuką pliuso(+) mygtuką. 
				2. Atsidariusiame lange spaudžiame  ,,ADD EXERCISE". 
				3. Iš pateikto sąrašo pasirenkame vieną iš sporto veiklų. 
				4. Atsidariusiame lange nurodome 30 minučių trukmę.
				5. Pasižymime kokią sudegintų kalorijų informacija mums pateikta. 
				6. Spaudžiame mygtuką ,,Add Exercise".
				8. Patikriname ar informacija išsisaugojo pagrindiniame ekrane 
				9. Pagrindiniame lange paspaudžiam ant sąrašo elemento sporto veiklos trukmės skaičiaus
				10. Pakeičiame elemento trukmę į 60 minučių
				11. Patikriname ar sporto veiklos elemento informacija pasikeitė korektiškai
				12. Patikriname ar bendra sudegintų kalorijų informacija pasikeitė korektiškai\\ \hline
        Tikėtasis rezultatas &  Pakeitus vieno iš sąrašo elemetų trukmę ji bus korektiškai pakeistas tiek sąrašo elemento informacijoje tiek ir bendroje sudegintų kalorijų informacijoje \\ \hline
        Gautas Rezultatas & Pakeitus vieno iš sąrašo elementų trukmę visa informacija pasikeitė korektiškai \\ \hline
        Statusas &  Atlikta\\ \hline
    \hline
    \end{tabular}
\end{center}

\iffalse XXXXXXXXXXXXXXXXXXXXXXXXXXXXXXXXXXXXXXXXXXXXXXXXXXXXXXXXXXXXXXXXXXXXXXXXXXXXXXXXXXXXXXXXXXXXXXXXXXXXXXXXXXXXXXXXXXXXXXXXXXXXXXXXXXXXXXX \fi

\begin{center}
    \begin{tabular}{ |p{5cm}|p{13cm}|}
    \hline
    	Testavimo atvejo numeris & CMT-2.3.1\\ \hline
    	Testavimo atvejis & Biometrikos pridėjimas \\ \hline
	Testavimo atvejo aprašymas & Ištestuoti ar pridedant naują biometriką jos reikšmė teisingai pridedama į bendrą sąrašą   \\ \hline
	Testavimo eiga & 1. Pagrindiniame aplikacijos lange spaudžiame mygtuką pliuso(+) mygtuką. 
				2. Atsidariusiame aplikacijos lange spaudžiame mygtuką ,,ADD BIOMETRIC". 
				3. Iš pateikto sąrašo pasirenkame vieną iš biometrikų. 
				4. Atsidariusiame lange nurodome 100 kaip biometrikos reikšmę.
				5. Spaudžiame mygtuką ,,ADD BIOMETRIC".
				6. Atsidarius pagrindiniam langui jame patikriname ar biometrinė informacija atitinka tą kurią nurodėme.\\ \hline
	Tikėtasis rezultatas &  Pridėjus naują biometriką pagrindiniame lange biometrikos informacija atitiks tą kuri buvome nurodę\\ \hline
	Gautas Rezultatas & Biometrika pridėta korektiškai  \\ \hline
	Statusas &  Atlikta\\ \hline
    \hline
    \end{tabular}
\end{center}

\begin{center}
    \begin{tabular}{ |p{5cm}|p{13cm}|}
    \hline
    	Testavimo atvejo numeris & CMT-2.3.2  \\ \hline
    	Testavimo atvejis & Biometrikos atvaizdavimas nurodant didelias reikšmes  \\ \hline
	Testavimo atvejo aprašymas & Ištestuoti ar pridedant naują biometriką ir nurodant jai didelę reikšmę programa duomenis pateiks ir atvaizduos korektiškai.  \\ \hline
	Testavimo eiga & 1. Pagrindiniame aplikacijos lange spaudžiame mygtuką pliuso(+) mygtuką. 
				2. Atsidariusiame aplikacijos lange spaudžiame mygtuką ,,ADD BIOMETRIC". 
				3. Iš pateikto sąrašo pasirenkame vieną iš biometrikų. 
				4. Atsidariusiame lange nurodome 2147483648(Int32 MAX + 1) reikšmę.
				5. Spaudžiame mygtuką ,,ADD BIOMETRIC".
				6. Atsidarius pagrindiniam langui jame patikriname ar biometrikos informacija atitinka tą kurią nurodėme.\\ \hline
	Tikėtasis rezultatas & Programa leis pridėti pasirinktą sporto veiklą su ilga trukme, informacija pagrindiniame lange bus pateikta ir nurodyta korektiškai  \\ \hline
	Gautas Rezultatas & Informacija išsaugoma korektiškai, ji pateikiama moksline notacija( 127E16 )  \\ \hline
	Statusas & Atlikta  \\ \hline
    \hline
    \end{tabular}
\end{center}

\begin{center}
    \begin{tabular}{ |p{5cm}|p{13cm}|}
    \hline
    	Testavimo atvejo numeris & CMT-2.3.3 \\ \hline
    	Testavimo atvejis &Biometrikos pridėjimas pateikiant neskaitinę informaciją \\ \hline
	Testavimo apvejo aprašymas & Ištestuoti ar bandant pridėti biometriką pateikiant neskaitinę jos reikšmę įvedimą(prie reikšmės pateikti ne sakaičiu o raides) programa neleis pridėti biometrikos  \\ \hline
	Testavimo eiga & 1. Pagrindiniame aplikacijos lange spaudžiame mygtuką pliuso(+) mygtuką. 
				2. Atsidariusiame aplikacijos lange spaudžiame mygtuką ,,ADD BIOMETRIC". 
				3. Iš pateikto sąrašo pasirenkame vieną iš biometrikų. 
				4. Prie biometrikos reikšmės įrašome ,,test" tekstinę eilutę. 
				5. Spaudžiame mygtuką ,,ADD BIOMETRIC".\\ \hline
	Tikėtasis rezultatas & Bandant pridėti biometriką jos reikšmę nurodant tekstu programa turi parodyti mums informacinį pranešimą ir neleisti mums pridėti biometrikos kol įvestis nebus pakeista į skaitinę įvestį \\ \hline
	Gautas Rezultatas & Programa neleidžia įrašyti biometrikai tekstinės reikės nes aplikacija pateikia tik skaitinę klaviatūrą \\ \hline
	Statusas & Atlikta \\ \hline
    \hline
    \end{tabular}
\end{center}

\begin{center}
    \begin{tabular}{ |p{5cm}|p{13cm}|}
    \hline
        Testavimo atvejo numeris & CMT-2.3.4  \\ \hline
        Testavimo atvejis & Biometrika nedingsta iš sąrašo įjungus ir vėl įjungus aplikaciją  \\ \hline
        Testavimo atvejo aprašymas & Ištestuoti ar išjungus ir vėl įjungus aplikaciją biometrikos sąrašo informacija lieka ta pati  \\ \hline
        Testavimo eiga &  1. Pagrindiniame aplikacijos lange spaudžiame mygtuką pliuso(+) mygtuką. 
				2. Atsidariusiame aplikacijos lange spaudžiame mygtuką ,,ADD BIOMETRIC". 
				3. Iš pateikto sąrašo pasirenkame vieną iš biometrikų. 
				4. Atsidariusiame lange nurodome 100 kaip biometrikos reikšmę.
				5. Spaudžiame mygtuką ,,ADD BIOMETRIC".
				6. Atsidarius pagrindiniam langui jame patikriname ar biometrinė informacija atitinka tą kurią nurodėme.
				7. Išjungiame ir vėl įjungiame aplikaciją
				8. Patikriname ar duomenys išliko tie patys\\ \hline
        Tikėtasis rezultatas &  Išjungus ir vėl įjungus aplikaciją biometrikos informacija pagrindiniame saraše biometrikos informacija nepasikeis\\ \hline
        Gautas Rezultatas & Biometrikos informacija po išjungimo ir įjungimo pateikiama korektiškai  \\ \hline
        Statusas & Atlikta  \\ \hline
    \hline
    \end{tabular}
\end{center}

\begin{center}
    \begin{tabular}{ |p{5cm}|p{13cm}|}
    \hline
        Testavimo atvejo numeris & CMT-2.3.5  \\ \hline
        Testavimo atvejis & Biometrikų galime pridėti tiek kiek norime  \\ \hline
        Testavimo atvejo aprašymas & Ištestuoti ar galima pridėti daugiau negu 100 skirtingų biometrikų į sąrašą ir pridėjus šias biometrikas informacija bus pateikta korektiškai \\ \hline
        Testavimo eiga &  1. Pagrindiniame aplikacijos lange spaudžiame mygtuką pliuso(+) mygtuką. 
				2. Atsidariusiame aplikacijos lange spaudžiame mygtuką ,,ADD BIOMETRIC". 
				3. Iš pateikto sąrašo pasirenkame vieną iš biometrikų. 
				4. Atsidariusiame lange nurodome 100 kaip biometrikos reikšmę.
				5. Spaudžiame mygtuką ,,ADD BIOMETRIC".
				6. Atsidarius pagrindiniam langui jame patikriname ar biometrinė informacija atitinka tą kurią nurodėme.
				7. Kartojame pirmjus 6 žingsnius 101 kartą \\ \hline
        Tikėtasis rezultatas & Pridėjus daugiau negu 100 skirtingų biometrikos tipų informacija bus pateikta korektiškai tiek atvaizdavimo atžvilgiu tiek ir informacijos korektiškumu.\\ \hline
        Gautas Rezultatas & Pridėjus 101 biometrikos vienetą informacija saraše pateikta ir nurodoma korektiškai  \\ \hline
        Statusas & Atlikta  \\ \hline
    \hline
    \end{tabular}
\end{center}


\begin{center}
    \begin{tabular}{ |p{5cm}|p{13cm}|}
    \hline
        Testavimo atvejo numeris &  CMT-2.3.6\\ \hline
        Testavimo atvejis & Galime pridėti tą pačią biometriką \\ \hline
        Testavimo atvejo aprašymas & Ištestuoti ar pridedant tą pačią biometriką prie sąrašo naują įrašą rodo kaip atskirą elementą saraše  \\ \hline
        Testavimo eiga &   1. Pagrindiniame aplikacijos lange spaudžiame mygtuką pliuso(+) mygtuką. 
				2. Atsidariusiame aplikacijos lange spaudžiame mygtuką ,,ADD BIOMETRIC". 
				3. Iš pateikto sąrašo pasirenkame vieną iš biometrikų. 
				4. Atsidariusiame lange nurodome 100 kaip biometrikos reikšmę.
				5. Spaudžiame mygtuką ,,ADD BIOMETRIC".
				6. Atsidarius pagrindiniam langui jame patikriname ar biometrinė informacija atitinka tą kurią nurodėme.
				7. Pakartojame pirmus šešis žingsnius\\ \hline
        Tikėtasis rezultatas &  Du kartus pridėjus tą pačią biometriką pagrindiniame saraše bus atvaizduoti du atskiri biometrikos elementai\\ \hline
        Gautas Rezultatas &  Pridėjus du identiškus biometrikos elementus į sąrašą jie saraše buvo atvaizduoti kaip du atskiri elementai korektiškai\\ \hline
        Statusas & Atlikta \\ \hline
    \hline
    \end{tabular}
\end{center}

\begin{center}
    \begin{tabular}{ |p{5cm}|p{13cm}|}
    \hline
        Testavimo atvejo numeris & CMT-2.3.7  \\ \hline
        Testavimo atvejis & Biometrikos reikšmės keitimas pagrindiniame lange \\ \hline
        Testavimo atvejo aprašymas & Ištestuoti ar pakeitusbiometrikos reikšmę pagrindiniame lange ji korektiškai pakeičiamas ir bendra informacija pasikeičia korektiškai \\ \hline
        Testavimo eiga &  1. Pagrindiniame aplikacijos lange spaudžiame mygtuką pliuso(+) mygtuką. 
				2. Atsidariusiame aplikacijos lange spaudžiame mygtuką ,,ADD BIOMETRIC". 
				3. Iš pateikto sąrašo pasirenkame vieną iš biometrikų. 
				4. Atsidariusiame lange nurodome 100 kaip biometrikos reikšmę.
				5. Spaudžiame mygtuką ,,ADD BIOMETRIC".
				6. Atsidarius pagrindiniam langui jame patikriname ar biometrinė informacija atitinka tą kurią nurodėme.
				7. Pagrindiniame lange spaudžiame ant biomterinės reikšmės saraše.
				8. Atsidaro biometrinės reikšmės keitimo langas
				9. Pakeičiame reikšmę
				10. Išsaugome
				11. Patikriname ar pagrindiniame lange reikšmė išsisaugo korektiškai \\ \hline
        Tikėtasis rezultatas &  Pakeitus vieno iš sąrašo elemetų reikšmę ji bus korektiškai pakeistas tiek sąrašo elemento informacijoje tiek ir bendroje biometrikos informacijoje \\ \hline
        Gautas Rezultatas & Pakeitus vieno iš sąrašo elementų reikšmę visa informacija pasikeitė korektiškai \\ \hline
        Statusas &  Atlikta\\ \hline
    \hline
    \end{tabular}
\end{center}

\iffalse XXXXXXXXXXXXXXXXXXXXXXXXXXXXXXXXXXXXXXXXXXXXXXXXXXXXXXXXXXXXXXXXXXXXXXXXXXXXXXXXXXXXXXXXXXXXXXXXXXXXXXXXXXXXXXXXXXXXXXXXXXXXXXXXXXXXXXX \fi
\pagebreak
\section{Defektų aprašas}
\begin{center}
    \begin{tabular}{ |p{1cm}| p{7cm} | p{7cm} | p{2cm} |}
    \hline
    Nr &  Funkcionalumas & Aprašymas & Sunkumas \\ \hline
    CMT-1.1.2 & Maisto informacijos atvaizdavimas dideliais kiekiais & Bandant pridėti maisto kiekius didesnius negu 999999 gramų programa pasirinktą kiekį autimatiškai padaro lygiu 999999g. Dėl per didelių skaičių būna iškraipomas programos pagrindinis langas ir informacija parodoma nekorektiškai & Lengvas \\ \hline
    CMT-1.1.3 & Maisto pridėjimas nurodant nekorektišką įvestį & Bandant pridėti maistą su tekstine kiekio įvestimi programa leidžia pridėti maistą automatiškai nurodydama 1 gramą kiekio & Lengvas \\ \hline
    CMT-1.2.2 & Sporto veiklos informacijos atvaizdavimas dideliais kiekiais & Bandant pridėti sporto veiklą kuri trunka ilgiau negu 999999 minutes programa pasirinktą kiekį automatiškai padaro lygiu 999999 minutėm. Dėl per didelių skaičių būna iškraipomas programos pagrindinis langas ir informacija parodoma nekorektiškai & Lengvas \\ \hline
    CMT-1.2.3 & Sporto veiklos pridėjimas nurodant nekorektišką įvestį & Bandant pridėti sporto veiklą su tekstine trukmės įvestimi programa leidžia pridėti sporto veiklą automatiškai nurodydama 30 minučių trukmę & Lengvas \\ \hline
    CMT-1.3.2 & Biometrikos informacijos atvaizdavimas dideliais kiekiais & Bandant pridėti biometrikos informaciją kurios reikšmė didesnė negu 999999 programa pasirinktą kiekį automatiškai padaro lygiu 999999. Dėl per didelių skaičių būna iškraipomas programos pagrindinis langas ir informacija parodoma nekorektiškai & Lengvas \\ \hline
   


    \hline
    \end{tabular}
\end{center}

\section{Išvada}

	







\end{document}

