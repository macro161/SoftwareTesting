\documentclass[oneside]{VUMIFPSkursinis}
\usepackage{algorithmicx}
\usepackage{algorithm}
\usepackage{algpseudocode}
\usepackage{amsfonts}
\usepackage{float}
\usepackage{amsmath}
\usepackage{bm}
\usepackage{caption}
\usepackage{color}
\usepackage{float}
\usepackage{graphicx}
\usepackage{listings}
\usepackage{subfig}
\usepackage{ltablex}
\usepackage{longtable}
\usepackage{wrapfig}
\usepackage{subfig}
\usepackage{pbox}
\renewcommand{\labelenumii}{\theenumii}
\renewcommand{\theenumii}{\theenumi.\arabic{enumii}.}
\renewcommand{\labelenumiii}{\theenumiii}
\renewcommand{\theenumiii}{\theenumii\arabic{enumiii}.}
\newcolumntype{P}[1]{>{\centering\arraybackslash}p{#1}}
\usepackage[%
	colorlinks=true,
	linkcolor=black
]{hyperref}
\university{Vilniaus universitetas}
\faculty{Matematikos ir informatikos fakultetas}
\department{Programų sistemų katedra}
\papertype{Programų sistemų testavimo laboratorinis darbas}
\title{Internetinės aplikacijos Cronometer testavimas}
\titleineng{Web appliaction Cronometer testing}
\status{3 kurso studentas}
\author{Matas Savickis}


\supervisor{Vytautas Valaitis, Asist., Dr.}
\date{Vilnius – \the\year}

\bibliography{bibliografija}
\begin{document}
\maketitle

\sectionnonum{Anotacija}
Šiuo darbu yra siekiama sukurti testavimo planą, testavimo atvejus ir atlikti testavimą www.cronometer.com  internetinei aplikacijai skirtai sekti maisto informaciją, sporto aktyvumą ir biologinius rodiklius.

\begin{itemize}
	\item{Matas Savickis - savickis.matas@gmail.com}
\end{itemize}

\tableofcontents

\section{Testavimo plano indentifikatoriai} CMT-1.1.1 - pirmas skaičius rodo ar testavimas buvo atliktas telefonu ar kompiuterius(1 - kompiuteris 2 - telefonas) antras skaičius rodo kelinta funkcionalumą testuosime, trečias skaičius rodo kelintą scenarijų testuojame
\section{Nuorodos}
	\begin{itemize}
		\item{https://jmpovedar.files.wordpress.com/2014/03/ieee-829.pdf}
		\item{http://pst.valaitis.net/PST}
	\end{itemize}
\section{Įvadas}
	Šis testavimo planas yra skirtas internetinės aplikacijos Cronometer testavimui. Pagrindinis testavimo tikslas bus įsitikinti ar su skirtingais įvesties duomenimis programa veikia korektiškai ir gražina korektiškus rezultatus. Bus testuojama maisto pridėjimas, sporto pridėjimas, biometrinių duomenų pridėjimas ir prisijungimas prie tinklalapio. Visas testavimas vyks rankiniu būdu.

\section{Testavimo objektai}
	\begin{enumerate}
		\item{www.cronometer.com}
	\end{enumerate}

\section{Rizikos}
	\begin{itemize}
		\item{Nėra programos dokumentacijos}
		\item{Jeigu kažką užlaušiu ant manęs pyks Cronometer kūrėjai}
	\end{itemize}

\section{Funkcionalumas kurį testuosime}
	\begin{enumerate}
		\item{Registravimasis į tinklalapi}
		\item{Prisijungimas prie tinklalapio}
		\item{Maisto vienetų pridėjimas sekimui, įvesties ir išvesties korektiškumas}
		\item{Sporto sekimo pridėjimas sekimui, įvesties ir išvesties korektiškumas}
		\item{Biometrinių duomenų pridėjimas į sistemą, įvesties ir išvesties korektiškumas}
	\end{enumerate}
\section{Funkcionalumas kurio netestuosime}
	\begin{enumerate}
		\item{Premium narystės funkcionalumus - kadangi testavimą atliks biednas studentas kuris neturi pinigų nusipirkti Premium narystės todėl šio funkcionalumo ir netestuosime}
		\item{Sistemos suderinamumą su išnamiaisiais įrenginiais sekimui(Fitbit) - priežastis ta pati kaip ir Premium narystės atveju}
		\item{Profilio funkcija - funkcionalumas nekritinis}
	\end{enumerate}	
\section{Strategija}
	\begin{itemize}
		\item{Specialūs įrankiai - nėra}
		\item{Apmokymas naudotis įrankiais - nėra}
		\item{Metrikos - bus sekama kiek laiko užtruko atlikti testavimą, suranstų klaidų skaičius, teisignai praeitų testų skaičius}
		\item{Metrikos bus sekamos atliekant rankinį testavimą}
		\item{Konfigūracijos valdymas - nėra}
		\item{Bus testuojama tik pradinė konfigūracija, kurią pateikia kūrėjai}
		\item{Bus naudojama stacionarus kompiuteris ir išmanusis telefonas}
		\item{Bus naudojama Google Chrome naršyklė abiem atvejais(PC ir telefonas)}
		\item{Regresiniai testai - nėra}
	\end{itemize}

\section{Sėkmės ir nesekmės kriterijai}
Testą laikysime pasisekusiu jeigu įvedus duomenis ar padarius veiksmą programa pateiks korektišką ir tikėtasi rezultatą ar neparodys klaidos pranešimų. Nepasisekusiu testu laikysime jeigu atlikus veiksą arba įvedus duomenis programos išvestis bus nenumatyta, netiksli, ne tai ko tikėjome arba programa gražins klaidos pranešimą.

\section{Testo nutraukimo criterijai}
Testavimą nutraukime tuo atveju jeigu su panašia įvestimi programa gražina tą patį rezultatą. Jeigu to pačio funkcionalumo varijacijoje(pvz. Maisto produktų dėjimas į sąrašą) klaida liks, laikysime, kad testas daugiau informacijos mums nebesuteiks ir testavimą nutrauksime įvertinti savo testavimo atvejams ir tolimesniam testavimui.

\section{Testų planai}
Planas skirtas rankiniam testavimui juodosios dežes metodu

\section{Likusios užduotys}
\begin{center}
\begin{tabular}{ |c|c|c| } 
 \hline
 Užduotis & Vykdytojas & Būsena \\ 
 Sukurti testavimo planą & Matas Savickis & Pabaigta \\ 
 Maisto pridėjimo testavimo atvejai  & Matas Savickis & Pabaigta \\ 
 Sporto pridėjimo testavimo atvejai & Matas Savickis & Pabaigta \\
 Biometrikos pridėjimo testavimo atvejai & Matas Savickis & Pabaigta \\
 \hline
\end{tabular}
\end{center}

\section{Aplinkos reikalavimai}
	\begin{itemize}
		\item{Turi veikti www.cronometer.com tinklalapios}
		\item{Reikalinga elektra}
		\item{Reikalingas interneto ryšys}
		\item{Atlikti testavimui reikia išmanioji telefono}
		\item{Atlikti testavimui reikia stacionaraus kompiuterio}
		\item{Norint pasiekti tiklalapį reikia modernios interneto naršyklės}
	\end{itemize}

\section{Personalas ir reikalingi apmokymai}
`	Testavimas bus atliekamas programų sistemų studentų. Komandoje iš viso puse vienas testuotojas. Svarbu kad testuotojas mokėtų naudotis kompiuterio pele bei klaviatūra. Taip pat svarbu, kad testuotojas neturėtų motorinių kurie keltų sunkumų naudotis telefono, pele ar klaviatūra. Taip pat yra svarbu, kad testuotojas mokėtų naudoti išmaniaisiais telefonais.

\section{Atsakomybės}
\begin{center}
\begin{tabular}{ |c|c| } 
 \hline
 Atsakomybė & Matas Savickis \\
 Rankinių testų kūrimas & X \\
 Rankinių testų vykdymas & X \\
 Testavimo plano sudarymas & X \\
 \hline
\end{tabular}
\end{center}

Už visą testavimo planavimo, sudarymo ir vykdymo procesą bus atskingas neapmokamas programų sistemų studentas Matas Savickis.  

\section{Tvarkaraštis}
Testavimo plano ir testavimo atvejų sudarymas ir testų atlikimas bus įvykdytas pagal gerai žinomą studentišką laiko valdymo metodą pavadinimu ,,viską padarysiu per vieną naktį prieš atsiskaitymą". Tai yra yra pats populiariausias darbo pasiskirstymo metodas, o kaip mes visi žinome tai kas populiaru yra gerai.\newline



Darbas bus atliekamas tokiais žingsniais:

\begin{enumerate}
	\item{Testavimo plano sudarymas}
	\item{Testavimo atvejų dokumento sudarymas}
	\item{Testavimo atvejų internetinei versijai sudarymas}
	\item{Testavimo atvejų mobiliai versijai sudarymas}
	\item{Internetinės versijos testų vykdymas}
	\item{Mobilios versijos testų vykdymas}
	\item{Rezultatų ir išvadų sudarymas}
\end{enumerate}

\section{Rizikos}
Testavimas gali būti nesėkmingas dėl testuotojų motyvacijos stokos. Kadangi darbas yra neapmokamas, darbai dažniausiai būna atidedami iki paskutinės minutės, kas gali sukelti problemų laiku spėjant pristatyti atliktus darbus. Rizika gali kirti ir dėl to, nes  testuotojas neturės sistemos dokumentacijos ir testo atvejai nebus pilni arba korektiški. Dėl dokumentacijos trūkumo gali nepavykti sugalvoti visų testavimo atvejų.

	





















\end{document}

